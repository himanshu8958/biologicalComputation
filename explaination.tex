\documentclass{article}
\usepackage{enumitem}
\usepackage{algpseudocode}
\usepackage{amsfonts}
\usepackage{subfiles}
\usepackage{xcolor}
\usepackage{txfonts}
\usepackage{amssymb}
\usepackage{graphicx}
\usepackage{float}
\usepackage{lmodern}
\usepackage{verbatim}
\usepackage{listings}

\newcommand\nat{\mathbb{N}}
\newcommand\eventually{\lozenge}
\newcommand\always{\square}
%% \newcommand\nextS{\lozenge}
\newcommand\nextS{\medcirc}
\newcommand\previousS{\circleddash}
\newcommand\implies{\Rightarrow}
\newcommand\andS{\wedge}
\newcommand\orS{\vee}
\newcommand\until{\mathit{U}}
\newcommand\weakUntil{\mathit{W}}
\newcommand\since{\mathit{S}}
\newcommand\someTimeInPast{\vartriangle}
\newcommand\muller{\mathcal{M}}
\newcommand\buchi{\mathcal{B}} 
\newcommand{\doesNotWork}[1]{{\color{gray} {#1}}}
\title{Biological Computation\\ Assignment 1}

\author{Himanshu Arora}

\begin{document}
\maketitle

\section*{Explanation of code}
The main work flow in Q1 is :
\begin{enumerate}
\item Generate all connected graphs (not digraphs) with n nodes.
\item Then I get the isomorphic equivalence classes from the previous step.
\item The graphs from the previous step are used to generate
  digraphs. If there is an edge $(u,v)$ is in the original graph, then
  there must exit 3 graphs, such that there is an edge from $u$ to $v$
  in one graph, an edge in the opposite direction in the second graph
  and in the third graph, both edges exist in the third graph.\\
  
  $\forall
  G, u, v. \exists G_1, G_2, G_3. (u, v) \in G \rightarrow (u, v) \in
  G_1 \wedge (v, u) \in G_2 \wedge (u, v) \in G_3 \wedge (v,u) \in G_3
  $.

\item The final step is to generate equivalence classes of graphs based
  on isomorphism.
  \begin{enumerate}
   \item I use the Weisfeiler Lehman Graph hash to generate a hash
     value for each graph. This hash value has the property that if 2
     graphs have different hash values they are guaranteed to be
     non-isomorphic. But in case their has value is identical there is
     a strong probability that they are isomorphic. 
   \item I use the hash value above to make buckets. Then within each
     bucket I check for isomorphism to confirm.
   \item This hashing technique reduced the number of isomorphic tests
     from quadratic to linear. 
  \end{enumerate}
  
\end{enumerate}

\section*{Declaration}
I did not use chat gpt to write the code but I took suggestion to
design the workflow and chatGpt mentioned Weisfeiler Lehman Graph hash
algorithm.

\section*{Question 1}

\begin{enumerate}[label=\alph*), start=2]
\item
\begin{figure}[hp]
  \verbatiminput{1.out}
  \caption{Q1, output for $n = 1$ }
\end{figure}

\begin{figure}[hp]
  \verbatiminput{2.out}
  \caption{Q1, output for $n = 2$ }
\end{figure}

For $n =3$ and $n= 4$, please see \texttt{3.out} and \texttt{4.out}.
\item  My implementation terminates for \texttt{n = 4} in an hour.
\item \texttt{n = 4} is the maximum termination that I have
  witnessed. For \texttt{ n = 5} the program does not terminate within
  10 hours. 
\end{enumerate}

\section*{Question 2}
I generate all the sub graphs of the given graph and then I find the
equivalence classes over the set of graphs based on isomorphism.


\end{document}

